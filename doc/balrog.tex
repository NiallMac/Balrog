\documentclass[12pt]{article}

\usepackage{amsfonts,amsmath,amssymb}
\usepackage{graphicx}
\usepackage{color}
\usepackage[svgnames]{xcolor}
\usepackage[colorlinks,linkcolor=blue,citecolor=blue,urlcolor=blue,pdftitle={Balrog},pdfauthor={Suchyta}]{hyperref}
\usepackage[margin=1.0in]{geometry}
\usepackage{title}


\renewcommand{\thesection}{\arabic{section}}
\renewcommand*\sectionautorefname{Section}
\renewcommand*\subsectionautorefname{Section}
\renewcommand*\subsubsectionautorefname{Section}

\newcommand{\balrog}{Balrog}
\newcommand{\opt}[1]{\texttt{--#1}}



\begin{document}
\balrogtitlepage

\newpage
\tableofcontents

\newpage
\section{Introduction}
\label{sec:intro}
\balrog{} is ...
%\autoref{sec:intro}

\section{Usage}
Usage

\section{Command Line Arguments}
To get a list of

\section{Output}
Each \balrog{} run generates a number of output files. 
These are organized into a fixed directory structure.
Users indicate the \opt{outdir} command line option, and
the remainder of the naming scheme occurs automatically,
placing files in subdirectories under \opt{outdir}.
Four subdirectories are written, labelled accoring to what
type of files they contain. 

\begin{itemize}
	\item \texttt{balrog\_cat}: Contains catalog files.
	\item \texttt{balorg\_image}: Contains image files.
	\item \texttt{balrog\_log}: Contains log files.
	\item \texttt{balrog\_sexconfig}: Contains files for configuring sextractor.
\end{itemize}

\end{document}
